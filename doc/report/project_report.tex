%%%%%%%%%%%%%%%%%%%%%%%%%%%%%%%%%%%%%%%%%
% Short Sectioned Assignment
% LaTeX Template
% Version 1.0 (5/5/12)
%
% This template has been downloaded from:
% http://www.LaTeXTemplates.com
%
% Original author:
% Frits Wenneker (http://www.howtotex.com)
%
% License:
% CC BY-NC-SA 3.0 (http://creativecommons.org/licenses/by-nc-sa/3.0/)
%
%%%%%%%%%%%%%%%%%%%%%%%%%%%%%%%%%%%%%%%%%

%----------------------------------------------------------------------------------------
%	PACKAGES AND OTHER DOCUMENT CONFIGURATIONS
%----------------------------------------------------------------------------------------

\documentclass[paper=letter, fontsize=11pt]{scrartcl} % A4 paper and 11pt font size

\usepackage[margin=1.0in]{geometry}

\usepackage[T1]{fontenc} % Use 8-bit encoding that has 256 glyphs
%\usepackage{fourier} % Use the Adobe Utopia font for the document - comment this line to return to the LaTeX default
\usepackage[english]{babel} % English language/hyphenation
\usepackage{amsmath,amsfonts,amsthm} % Math packages

\usepackage{sectsty} % Allows customizing section commands
%\allsectionsfont{\centering\normalfont\scshape} % Make all sections centered, the default font and small caps
%\allsectionsfont{\scshape} % Make all sections centered, the default font and small caps

\usepackage{fancyhdr} % Custom headers and footers
\pagestyle{fancyplain} % Makes all pages in the document conform to the custom headers and footers
\fancyhead{} % No page header - if you want one, create it in the same way as the footers below
\fancyfoot[L]{} % Empty left footer
\fancyfoot[C]{} % Empty center footer
\fancyfoot[R]{\thepage} % Page numbering for right footer
\renewcommand{\headrulewidth}{0pt} % Remove header underlines
\renewcommand{\footrulewidth}{0pt} % Remove footer underlines
\setlength{\headheight}{13.6pt} % Customize the height of the header

\numberwithin{equation}{section} % Number equations within sections (i.e. 1.1, 1.2, 2.1, 2.2 instead of 1, 2, 3, 4)
\numberwithin{figure}{section} % Number figures within sections (i.e. 1.1, 1.2, 2.1, 2.2 instead of 1, 2, 3, 4)
\numberwithin{table}{section} % Number tables within sections (i.e. 1.1, 1.2, 2.1, 2.2 instead of 1, 2, 3, 4)

%\setlength\parindent{0pt} % Removes all indentation from paragraphs - comment this line for an assignment with lots of text

% For code blocks.
\usepackage{listings}

% For figures.
\usepackage{graphicx}
\usepackage{subcaption}

% For subsection formatting.
\renewcommand{\thesubsection}{(\alph{subsection})}

\usepackage{titlesec}
%\titleformat{\subsection}[runin]{\normalfont\large}{\thesubsection}{1em}{}

%----------------------------------------------------------------------------------------
%	TITLE SECTION
%----------------------------------------------------------------------------------------

\newcommand{\horrule}[1]{\rule{\linewidth}{#1}} % Create horizontal rule command with 1 argument of height

\title{
\normalfont \normalsize
%\textsc{university, school or department name} \\ [25pt] % Your university, school and/or department name(s)
\horrule{0.5pt} \\[0.4cm] % Thin top horizontal rule
\huge MIT 16.31 - Project Report \\ % The assignment title
\horrule{2pt} \\[0.5cm] % Thick bottom horizontal rule
}

\author{W. Nicholas Greene} % Your name

\date{\normalsize2014.12.05} % Today's date or a custom date

\begin{document}

\maketitle % Print the title

%----------------------------------------------------------------------------------------
% Introduction
%----------------------------------------------------------------------------------------
\section{Introduction}
Autonomous navigation and control of micro-aerial vehicles (MAVs) is an increasingly popular
subject of academic research, with several key advances occurring over the last ten years. 
Small, four-rotor craft (\textit{quadrotors}) are desirable platforms for this kind of 
robotics research due to their ability to carry large sensor payloads and maintain stability 
in the hover regime. There is significant interest, however, in having these types of vehicles 
execute aggressive trajectories far from the hover state (e.g. flying quickly through 
cluttered environments such as buildings and forests).

This project focuses on the implementation of a modern, nonlinear controller for stabilizing the
3D position and yaw angle of a quadrotor far from the hover regime based on \cite{lee2010geometric}. 
I implemented a modified version of this controller in Python and validated the algorithm 
in simulation using the \texttt{hector\_quadrotor} simulator - a software package that simulates the
dynamics of a quadrotor using the Robot Operating System (ROS) and the Gazebo robot
simulation toolkit \cite{2012simpar_meyer, quigley2009ros, koenig2004design}. Results
show that the controller is able to successfully track aggressive trajectories with 
significant linear and angular accelerations.

%----------------------------------------------------------------------------------------
% Geometric Control on SE3
%----------------------------------------------------------------------------------------
\section{Geometric Control on SE(3)}
Controlling a quadrotor's six degrees of freedom (3D position and 3D orientation) is a
non-trivial task for several reasons. First, the dynamics involved are non-linear, which
makes linear control schemes  such as proportional-integral-derivative (PID) controllers or 
linear quadratic regulators (LQR) less attractive. Second, the quadrotor system state
does not lie in Euclidean space, but on the special Euclidean group SE(3) - the manifold composed
of 3D positions and 3D orientations ($\textrm{SE}(3) = \{\mathbb{R}^3 \times \textrm{SO}(3)\}$
where SO(3) is the special orthogonal group representing 3D rotations).
Finally, the system can only be controlled via the forces and torques produced by its
four rotors. (While this arrangement is beneficial from a modelling complexity standpoint, 
it requires the interactions of all the rotors to effectively control the vehicle).

\subsection{Dynamics Model}
The equations governing the quadrotor's dynamics follow directly from Newton's second law
of motion:
\begin{align}
\dot{x} &= v \\
m \dot{v} &= m g e_3 - f R e_3 \\
\dot{R} &= R \hat{\Omega} \\
M &= J \hat{\Omega} + \Omega \times J\Omega
\end{align}
where $x$ and $v$ are the system's position and velocity, $R$ and $\Omega$
are the orientation (represented as a rotation matrix) and angular velocity (in the 
body frame), $m$ and $J$ are the mass and intertia matrix (in the body frame), $f$ is the
total thrust from the rotors, $g$
is the gravitational acceleration constant, $e_3$ is the downward pointing unit vector
in the standard (Right-Foward-Down) basis, and $\hat{\cdot}: \mathbb{R}^3 \rightarrow \textrm{so}(3)$
is the \textit{hat} operator, which maps vectors in $\mathbb{R}^3$ to the set of $3 \times 3$
skew-symmetric matrices (i.e. the Lie algebra so(3) associated with SO(3)).

\subsection{Tracking Errors}
\subsection{Tracking Controller}

%----------------------------------------------------------------------------------------
% Implementation
%----------------------------------------------------------------------------------------
\section{Implementation}

%----------------------------------------------------------------------------------------
% Results
%----------------------------------------------------------------------------------------
\section{Results}
\subsection{Circle Trajectory}

\subsection{Lissajous Trajectory}


%----------------------------------------------------------------------------------------
% Bibliography
%----------------------------------------------------------------------------------------
\bibliography{references}{}
\bibliographystyle{plain}

%---------%% -------------------------------------------------------------------------------
%% %	PROBLEM 1
%% %----------------------------------------------------------------------------------------
%% \section{}
%% \subsection{}
%% For $G_c(s)G_p(s) = \frac{K}{(s^2 + 6s + 8)(s^2 + 2s + 5)}$, there are no open loop zeros
%% and four open loop poles since the denominator of $G_c(s)G_p(s)$ is a fourth order polynomial.
%% Solving for the roots of the denominator we have open loop poles at $\{-2, -4, -1 + 2j, -1 - 2j\}$.
%% The closed loop poles on the root locus will start at the open loop poles for $K = 0$, and
%% then converge to open loop zeros. Since there are no open loop zeros, the closed loop poles
%% will diverge to infinity with asymptotes at $\frac{\pi \pm 2\pi i}{4} = \pi/4 \pm (\pi/2) i$
%% for $i = 0, 1, \ldots$. The
%% origin of the asymptotes will lie at $(-2 - 4 + (-1 + 2j) + (-1 - 2j))/4 = -2$. Finally,
%% since there are an odd number of poles/zeros to the right of the the pole at $-4$,
%% the real-line between $-4$ and $-2$ will also reside on the locus, resulting in
%% following figure:
%% \begin{figure}[h!]
%%   \centering
%%   \includegraphics[width=0.6\textwidth]{1a_rlocus}
%% %  \caption{New views after reconstructing scene depths with Simple World assumption.}
%% \end{figure}

%% \subsection{}
%% For $G_c(s)G_p(s) = \frac{K(s^2 + 4.5s + 5.625)}{s(s+1)(s+2)}$, we have two open loop zeros
%% at $\{-2.25 + 0.75j, -2.25 - 0.75j\}$ and three open loop poles at
%% $\{0, -2, -1\}$. Using the real-line rule, the real line to the left of $-2$ will lie on
%% the locus, as well as the
%% segment between $-1$ and $0$. Since we have one more pole than zeros, we will have one
%% asymptote at $\pi$, satisfied by the pole at $-2$.. The poles at $-1$ and $0$ will move
%% closer together on the real axis before converging to the complex zeros resulting in:
%% \begin{figure}[h!]
%%   \centering
%%   \includegraphics[width=0.6\textwidth]{1b_rlocus}
%% %  \caption{New views after reconstructing scene depths with Simple World assumption.}
%% \end{figure}

%% \subsection{}
%% For $G_c(s)G_p(s) = \frac{K(s+5)}{s^4 - 9s^2}$, we will have one open loop zero at $-5$
%% and four open loop poles at $\{0, 0, -3, 3\}$. Since we have three more poles than zeros,
%% the root locus will have asymptotes at $\pi/3 + (2\pi/3) i$ for $i = 0, 1, \ldots$. Using the
%% real-line rule, the segments from $[-3, 0]$ and $[0, 3]$ will lie on the locus. One of the poles
%% at $0$ and the pole at $3$ will converge together before breaking away on asymptotes at $\pi/3$ and
%% $-\pi/3$. The other pole at $0$ will converge with $-3$. When they meet, one will split off
%% towards the open loop pole at $-5$, while the other continues on the real axis towards
%% negative infinity.
%% \begin{figure}[h!]
%%   \centering
%%   \includegraphics[width=0.6\textwidth]{1c_rlocus}
%% %  \caption{New views after reconstructing scene depths with Simple World assumption.}
%% \end{figure}

%% %----------------------------------------------------------------------------------------
%% %	PROBLEM 2
%% %----------------------------------------------------------------------------------------
%% \pagebreak
%% \section{}
%% \subsection{}
%% To plot the root locus, we must first compute the closed loop transfer function. By following
%% the signal from the output to the input, we arrive at the following relation:
%% \begin{align*}
%% \theta(s) &= G_p(s) [K_p(\theta_c(s) - \theta(s)) - sK_d \theta(s)].
%% \end{align*}
%% After rearranging terms we arrive at the closed loop transfer function
%% \begin{align*}
%% \frac{\theta(s)}{\theta_c(s)} &= \frac{K_pG_p(s)}{1 + K_pG_p(s) + sK_dG_p(s)}.
%% \end{align*}

%% With $K_d = 0$, the denominator of the transfer function becomes
%% \begin{align*}
%% 1 + K_pG_p(s)
%% \end{align*}
%% where we see that the open loop system is just $G_p(s) = \frac{0.9}{(s^2 - 0.03)}$ with no
%% zeros and two poles at $\pm \sqrt{0.03}$. With more poles than zeros, the closed loop poles
%% will tend towards asymptotes at $\pi/2 \pm \pi i$ for $i = 0, 1, \ldots$. The origin of the
%% asymptotes will lie at $0$ and the segment between $\pm \sqrt{0.03}$ will also lie on the
%% locus. Thus, we have the following behavior:
%% \begin{figure}[h!]
%%   \centering
%%   \includegraphics[width=0.6\textwidth]{2a_rlocus}
%% %  \caption{New views after reconstructing scene depths with Simple World assumption.}
%% \end{figure}

%% We see here the the system is unstable when $K$ is small (the pole at $+\sqrt{0.03}$)
%% and becomes marginally stable when $K$ is large, but oscillatory
%% (the convergence to the imaginary axis).
%% At no point is the system strictly stable, no matter the value of $K$.

%% \subsection{}
%% We wish to set $K_p$ and $K_d$ such that the closed loop poles lie at $s = -0.2 \pm 0.3j$.
%% By manipulating the denominator of the transfer function we have
%% \begin{align*}
%% 1 + K_pG_p(s) + sK_dG_p(s) &= 1 + K_p \frac{0.9}{s^2 - 0.03} + sK_d\frac{0.9}{s^2 - 0.03}.
%% \end{align*}
%% Multiplying through by $s^2 - 0.03$ and setting to zero yields the following constraint(s)
%% that $K_p$ and $K_d$ must satisfy when $s = -0.2 \pm 0.3j$:
%% \begin{align*}
%% s^2 + 0.9K_ds + 0.9K_p - 0.03 = 0.
%% \end{align*}
%% We note that with the relationship $(s - p_1)(s - p_2) = s^2 - s(p_1 + p_2) + p_1p_2$, we see
%% that
%% \begin{align*}
%% 0.9K_d &= - (-0.2 + 03j + -0.2 - 0.3j) \\
%% 0.9K_p - 0.03 &= (-0.2 + 0.3j)(-0.2 - 0.3j),
%% \end{align*}
%% which gives us $K_p = 8/45$ and $K_d = 4/9$.

%% \subsection{}
%% With $K_d = 4/9$, the root locus becomes
%% \begin{figure}[h!]
%%   \centering
%%   \includegraphics[width=0.6\textwidth]{2c_rlocus}
%% %  \caption{New views after reconstructing scene depths with Simple World assumption.}
%% \end{figure}

%% from which we can see that the system is stable for appropriate values of $K_p > 0$.

%% \subsection{}
%% With $K_d = 4/9$ and $K_p = 1$, the impulse response of the closed loop system is
%% \begin{align*}
%% h(t) &= (9/\sqrt{83}) \exp(-t/5) \sin(\sqrt{83}t/10)
%% \end{align*}
%% with the following behavior:
%% \begin{figure}[h!]
%%   \centering
%%   \includegraphics[width=0.55\textwidth]{2d_impulse}
%% %  \caption{New views after reconstructing scene depths with Simple World assumption.}
%% \end{figure}

%% %----------------------------------------------------------------------------------------
%% %	PROBLEM 3
%% %----------------------------------------------------------------------------------------
%% \section{}
%% To find the range of $\lambda$ for which the system is stable for all $K > 0$, we first find
%% the closed loop tranfer function. Considering the inner velocity control loop we have
%% \begin{align*}
%% V(s) &= (\frac{10}{s+11})(\frac{K}{s + \lambda})(Z(s) - V(s))
%% \end{align*}
%% where $Z(s)$ is the velocity error. Solving for $V(s)/Z(s)$ yields
%% \begin{align*}
%% \frac{V(s)}{Z(s)} &= \frac{(\frac{10}{s+11})(\frac{K}{s + \lambda})}{1 + (\frac{10}{s+11})(\frac{K}{s + \lambda})} \\
%% &= \frac{10K}{(s+11)(s+\lambda) + 10K}.
%% \end{align*}

%% The outer loop then must satisfy
%% \begin{align*}
%% Y(s) &= \frac{1}{s}\frac{V(s)}{Z(s)}(R(s) - Y(s)
%% \end{align*}
%% such that $Y(s)/R(s)$ is given by
%% \begin{align*}
%% \frac{Y(s)}{R(s)} &= \frac{\frac{1}{s}\frac{V(s)}{Z(s)}}{1 + \frac{1}{s}\frac{V(s)}{Z(s)}}.
%% \end{align*}

%% We must now rearrange the denominator of $Y(s)/R(s)$ into the form $1 + KL(s)$ in order to apply
%% the root locus methods. We therefore have
%% \begin{align*}
%% 1 + \frac{1}{s}\frac{V(s)}{Z(s)} &= 0 \\
%% &= 1 + \frac{1}{s}\left(\frac{10K}{(s+11)(s+\lambda) + 10K}\right) \\
%% &= s + \frac{10K}{(s+11)(s+\lambda) + 10K} \\
%% &= s(s+11)(s+\lambda) + s10K + 10K \\
%% &= s(s+11)(s+\lambda) + K(s10 + 10) \\
%% &= 1 + K \frac{10(s + 1)}{s(s+11)(s+\lambda)},
%% \end{align*}
%% from which we can see that $L(s) = \frac{10(s + 1)}{s(s+11)(s+\lambda)}$.

%% The open loop system has one zero at $-1$ and three poles at $\{0, -11, -\lambda\}$.
%% One of the poles will converge to the zero at $-1$, while the other two will trend towards
%% asymptotes at $\pi/2 \pm \pi i$ for $i = 0, 1, \ldots$ with origin at $O = (-11 - \lambda + 1)/2 = -(10 + \lambda)/2$.
%% For the system to be stable for all $K > 0$, all of the open loop poles must have
%% a negative real component less than or equal zero so that the system will be stable when
%% $K$ is small. This puts the constraint that $\lambda \geq 0$. For the system to be stable
%% when $K$ is large, we must place $O$ such that it is strictly less than zero yielding
%% $-(10 + \lambda)/2 < 0 \implies \lambda > -10$. The combination of these constraints leads to
%% $\lambda \geq 0$.

%% For $\lambda = 1$, the roots locus is the following:
%% \begin{figure}[h!]
%%   \centering
%%   \includegraphics[width=0.6\textwidth]{3_rlocus}
%% %  \caption{New views after reconstructing scene depths with Simple World assumption.}
%% \end{figure}

%% %----------------------------------------------------------------------------------------
%% %	PROBLEM 4
%% %----------------------------------------------------------------------------------------
%% \section{}
%% \subsection{}
%% The closed loop transfer function of the system in question is
%% \begin{align*}
%% \frac{Y(s)}{U(s)} &= \frac{G_p(s)G_c(s)}{1 + G_p(s)G_c(s)},
%% \end{align*}
%% where the denominator expands to
%% \begin{align*}
%% 1 + G_p(s)G_c(s) &= 1 + K(\frac{s-2}{s})(\frac{s+z}{s+p}).
%% \end{align*}
%% From here we can see that the system has two open loop zeros at $\{2, -z\}$ and two open loop
%% poles at $\{0, -p\}$.

%% From the root locus rules, we know that the closed loop poles will start at the open loop poles
%% when $K$ is small and converge to the open loop zeros when $K$ is large. In order for the
%% system to be strictly stable for some range of $K$, we must ``pull'' the pole at $0$ away
%% from the zero at $2$, into the negative-real half of the complex plane, while also making
%% sure the pole at $-p$ lies in the negative-real half as well. We can do this
%% by making $p > 0$ and  making sure there are an odd number of poles/zeros
%% to the right of $-p$ by making $z < 0$. Choosing $p = 1$ and $z = -1$, the root locus is given
%% below and we can see that the system is strictly stable for some range of $K$.
%% \begin{figure}[h!]
%%   \centering
%%   \includegraphics[width=0.6\textwidth]{4a_rlocus}
%% %  \caption{New views after reconstructing scene depths with Simple World assumption.}
%% \end{figure}

%% To find this range, we will substitute $s = j\alpha$ into
%% the characteristic polynomial (representing the crossing of the imaginary axis) and solve for $K$.
%% With $p = 1$ and $z = -1$, the characteristic polynomial is
%% \begin{align*}
%% \phi(s) &= 1 + K (\frac{s - 2}{s}) (\frac{s - 1}{s + 1}) \\
%% &= 0.
%% \end{align*}
%% Solving for $K$, we have $K = \frac{-s(s + 1)}{(s-2)(s-1)}$. We substitute $s = j\alpha$ to obtain
%% \begin{align*}
%% K &= \frac{-j\alpha(j\alpha + 1)}{(j\alpha - 2)(j\alpha - 1)} \\
%% &= \frac{\alpha^2 - j \alpha}{(2-\alpha^2) - j3\alpha}.
%% \end{align*}
%% We note that complex division is defined as
%% \begin{align*}
%% \frac{a + jb}{c + jd} &= \frac{ac + bd}{c^2 + d^2} + j\frac{bc - ad}{c^2 + d^2}
%% \end{align*}
%% and apply this definition to the expression for $K$ with $a = \alpha^2$, $b = -\alpha$,
%% $c = 2 - \alpha^2$, and $d = -3\alpha$ yielding
%% \begin{align*}
%% K &= \frac{\alpha^2(5 - \alpha^2)}{(2-\alpha^2)^2 + 9\alpha^2}
%%      + j\frac{\alpha(-2 + 4\alpha^2)}{(2-\alpha^2)^2 + 9\alpha^2}.
%% \end{align*}

%% We know that $K$ must be real and positive, so we set $\Im(K) = 0$, which reduces to
%% \begin{align*}
%% \alpha(-2 + 4\alpha^2) &= 0,
%% \end{align*}
%% from which we can see that $\alpha = \{0, \pm 1/\sqrt{2}\}$. The solution $\alpha = 0$
%% represents the locus crossing the origin at $K = 0$, while $\alpha = \pm 1/\sqrt{2}$ represents
%% the crossing of the imaginary axis at $K > 0$. We substitute these solutions into the
%% expression for $K$ and and find that the root locus crosses the imaginary axis at $K = 1/3$.
%% Thus, for the system to be strictly stable, $0 < K < 1/3$.

%% \subsection{}
%% We must now pick a specific $K$ such that the system is strictly stable, the damping ratio $\zeta$
%% is between $0.4$ and $0.6$, and the natural frequency $w_n$ is minimized. Inspecting the
%% root locus plot, we see that the natural frequncy $w_n$ will be minimized when the
%% damping ratio $\zeta$ is maximized. Thus, we set $\zeta = 0.6$ and solve for $w_n$ to find
%% $K$.

%% The relationship between the complex $s$ variable and $w_n$ and $\zeta$ is given by
%% \begin{align*}
%% s &= -\zeta w_n + j w_n \sqrt{1 - \zeta^2} \\
%% &= w_n(-0.6 + j 0.8).
%% \end{align*}
%% We substitute this expression into that for $K$ to obtain
%% \begin{align*}
%% K &= \frac{-s(s + 1)}{(s-2)(s-1)} \\
%% &= \frac{(0.28 w_n^2 + 0.6 w_n) + j (0.96 w_n^2 - 0.8 w_n)}{(-0.28 w_n^2 + 1.8 w_n + 2) + j(-0.96 w_n^2 - 2.4 w_n)}.
%% \end{align*}
%% We apply the same procedure as before, applying the complex division operation and
%% setting the imaginary component of the quotient to zero, yielding the following
%% constraint:
%% \begin{align*}
%% \Im(K) &= \eta w_n(w_n^2 + 0.6 w_n - 0.5) \\
%% &= 0,
%% \end{align*}
%% where $\eta$ is the normalizer. Solving the above equation results in three solutions,
%% $w_n = \{-1.068, 0, 0.468\}$. We take the positive solution and find that the required
%% value of $K$ is $K = 0.123$.

%% %----------------------------------------------------------------------------------------
%% %	PROBLEM 5
%% %----------------------------------------------------------------------------------------
%% \section{}
%% I spent roughly 10 hours on this problem set, including typesetting. 

\end{document}
